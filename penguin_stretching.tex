\PassOptionsToPackage{unicode=true}{hyperref} % options for packages loaded elsewhere
\PassOptionsToPackage{hyphens}{url}
%
\documentclass[]{article}
\usepackage{lmodern}
\usepackage{amssymb,amsmath}
\usepackage{ifxetex,ifluatex}
\usepackage{fixltx2e} % provides \textsubscript
\ifnum 0\ifxetex 1\fi\ifluatex 1\fi=0 % if pdftex
  \usepackage[T1]{fontenc}
  \usepackage[utf8]{inputenc}
  \usepackage{textcomp} % provides euro and other symbols
\else % if luatex or xelatex
  \usepackage{unicode-math}
  \defaultfontfeatures{Ligatures=TeX,Scale=MatchLowercase}
\fi
% use upquote if available, for straight quotes in verbatim environments
\IfFileExists{upquote.sty}{\usepackage{upquote}}{}
% use microtype if available
\IfFileExists{microtype.sty}{%
\usepackage[]{microtype}
\UseMicrotypeSet[protrusion]{basicmath} % disable protrusion for tt fonts
}{}
\IfFileExists{parskip.sty}{%
\usepackage{parskip}
}{% else
\setlength{\parindent}{0pt}
\setlength{\parskip}{6pt plus 2pt minus 1pt}
}
\usepackage{hyperref}
\hypersetup{
            pdftitle={Penguin Project},
            pdfauthor={May and Caitlin},
            pdfborder={0 0 0},
            breaklinks=true}
\urlstyle{same}  % don't use monospace font for urls
\usepackage[margin=1in]{geometry}
\usepackage{longtable,booktabs}
% Fix footnotes in tables (requires footnote package)
\IfFileExists{footnote.sty}{\usepackage{footnote}\makesavenoteenv{longtable}}{}
\usepackage{graphicx,grffile}
\makeatletter
\def\maxwidth{\ifdim\Gin@nat@width>\linewidth\linewidth\else\Gin@nat@width\fi}
\def\maxheight{\ifdim\Gin@nat@height>\textheight\textheight\else\Gin@nat@height\fi}
\makeatother
% Scale images if necessary, so that they will not overflow the page
% margins by default, and it is still possible to overwrite the defaults
% using explicit options in \includegraphics[width, height, ...]{}
\setkeys{Gin}{width=\maxwidth,height=\maxheight,keepaspectratio}
\setlength{\emergencystretch}{3em}  % prevent overfull lines
\providecommand{\tightlist}{%
  \setlength{\itemsep}{0pt}\setlength{\parskip}{0pt}}
\setcounter{secnumdepth}{0}
% Redefines (sub)paragraphs to behave more like sections
\ifx\paragraph\undefined\else
\let\oldparagraph\paragraph
\renewcommand{\paragraph}[1]{\oldparagraph{#1}\mbox{}}
\fi
\ifx\subparagraph\undefined\else
\let\oldsubparagraph\subparagraph
\renewcommand{\subparagraph}[1]{\oldsubparagraph{#1}\mbox{}}
\fi

% set default figure placement to htbp
\makeatletter
\def\fps@figure{htbp}
\makeatother


\title{Penguin Project}
\author{May and Caitlin}
\date{4/20/2020}

\begin{document}
\maketitle

import datasets

This document includes - summary stats: mean median interstretch
interval

\begin{itemize}
\item
  Summary stats for each penguin group size
\item
  Plot of the average interstretch interval per group-size
\item
  Plot of the average interstretch interval per group-size per
  individual(*group size)
\item
  regressions of slope and fit for these data
\item
  Full histogram of interstretch intervals for all data
\item
  Histogram of interstretch intervals seperated by group size
\item
  Shapiro- wilks and KS tests of normality
\end{itemize}

OTHER interesting data visualized:

\begin{itemize}
\item
  When did people observe?
\item
  Did average stretch rate vary by time of day?
\item ~
  \hypertarget{did-group-size-vary-by-time-of-day}{%
  \subsection{Did group size vary by time of
  day?}\label{did-group-size-vary-by-time-of-day}}
\end{itemize}

Summary stretching stats:

Mean interstretch interval:

\begin{verbatim}
## [1] 132.2264
\end{verbatim}

median interstretch interval:

\begin{verbatim}
## [1] 64
\end{verbatim}

Summary data by group size:

n\_penguins = group size stretch\_mean = inerstretch interval mean
stretch\_sd = interstretch interval standard deviation stretch\_median =
median interstretch interval n\_obs = number of observations / group

\begin{longtable}[]{@{}rrrrrrr@{}}
\caption{Stretching by group size}\tabularnewline
\toprule
n\_penguins & stretch\_mean & stretch\_sd & stretch\_median & n\_obs &
avg\_interstretch\_per\_individual &
med\_interstretch\_per\_individual\tabularnewline
\midrule
\endfirsthead
\toprule
n\_penguins & stretch\_mean & stretch\_sd & stretch\_median & n\_obs &
avg\_interstretch\_per\_individual &
med\_interstretch\_per\_individual\tabularnewline
\midrule
\endhead
1 & 272.70500 & 361.4757 & 162.0 & 200 & 272.7050 & 162.0\tabularnewline
2 & 198.54874 & 258.3485 & 108.0 & 636 & 397.0975 & 216.0\tabularnewline
3 & 158.04665 & 226.8575 & 77.5 & 986 & 474.1400 & 232.5\tabularnewline
4 & 136.23652 & 214.5862 & 68.0 & 1150 & 544.9461 & 272.0\tabularnewline
5 & 107.95697 & 198.8644 & 54.0 & 1255 & 539.7849 & 270.0\tabularnewline
6 & 123.49203 & 247.7682 & 53.0 & 878 & 740.9522 & 318.0\tabularnewline
7 & 111.83460 & 165.8954 & 50.0 & 659 & 782.8422 & 350.0\tabularnewline
8 & 101.48911 & 172.6994 & 54.0 & 597 & 811.9129 & 432.0\tabularnewline
9 & 92.78958 & 118.8590 & 53.0 & 499 & 835.1062 & 477.0\tabularnewline
10 & 98.40244 & 135.1804 & 51.5 & 246 & 984.0244 & 515.0\tabularnewline
11 & 101.95918 & 114.1070 & 59.5 & 98 & 1121.5510 & 654.5\tabularnewline
12 & 147.05556 & 141.9295 & 77.5 & 36 & 1764.6667 & 930.0\tabularnewline
\#\#\# Inter-st & retch interval & by group size & & & &\tabularnewline
\includegraphics{\#section-3}(penguin\_s & tretching\_files &
/figure-latex & /unnamed-chunk-7- & 1.pdf)\textless{}! & --
--\textgreater{} &\tabularnewline
\bottomrule
\end{longtable}

\begin{verbatim}
## 
## Call:
## lm(formula = penguin_group$inter_stretch_intervals_s ~ penguin_group$n_penguins)
## 
## Residuals:
##    Min     1Q Median     3Q    Max 
## -185.8 -105.6  -62.0   21.3 3783.4 
## 
## Coefficients:
##                          Estimate Std. Error t value Pr(>|t|)    
## (Intercept)               198.287      6.080   32.61   <2e-16 ***
## penguin_group$n_penguins  -12.448      1.044  -11.92   <2e-16 ***
## ---
## Signif. codes:  0 '***' 0.001 '**' 0.01 '*' 0.05 '.' 0.1 ' ' 1
## 
## Residual standard error: 213.1 on 7238 degrees of freedom
## Multiple R-squared:  0.01926,    Adjusted R-squared:  0.01913 
## F-statistic: 142.2 on 1 and 7238 DF,  p-value: < 2.2e-16
\end{verbatim}

Basically, the inter-stretch interval decreases by about 12 seconds for
every increase of 1 penguin in a group

But of course, if you have more penguins around, the interstretch
interval will go down even if each penguin is just stretching at the
same rate. So we can also account for the \# of individuals around by
multiplying the interstretch interval by the \# of penguins around

\#\#\#Interstretch interval per individual group size/ individual.
\includegraphics{penguin_stretching_files/figure-latex/unnamed-chunk-8-1.pdf}

\begin{verbatim}
## 
## Call:
## lm(formula = penguin_group$is_p_group ~ penguin_group$n_penguins)
## 
## Residuals:
##     Min      1Q  Median      3Q     Max 
## -1081.1  -485.0  -276.7    93.4 22759.0 
## 
## Coefficients:
##                          Estimate Std. Error t value Pr(>|t|)    
## (Intercept)               224.967     30.003   7.498 7.25e-14 ***
## penguin_group$n_penguins   76.343      5.152  14.819  < 2e-16 ***
## ---
## Signif. codes:  0 '***' 0.001 '**' 0.01 '*' 0.05 '.' 0.1 ' ' 1
## 
## Residual standard error: 1051 on 7238 degrees of freedom
## Multiple R-squared:  0.02945,    Adjusted R-squared:  0.02931 
## F-statistic: 219.6 on 1 and 7238 DF,  p-value: < 2.2e-16
\end{verbatim}

\includegraphics{penguin_stretching_files/figure-latex/unnamed-chunk-8-2.pdf}

\begin{verbatim}
As group size gets bigger, the time between stretches (per individual) increases by 75s / group member
number of observations of stretching at each group size. 
![](penguin_stretching_files/figure-latex/unnamed-chunk-9-1.pdf)<!-- --> 


Histograms 
![](penguin_stretching_files/figure-latex/unnamed-chunk-10-1.pdf)<!-- --> ![](penguin_stretching_files/figure-latex/unnamed-chunk-10-2.pdf)<!-- --> 

Stats: is this distribution normal?
\end{verbatim}

\hypertarget{section}{%
\subsection{}\label{section}}

\hypertarget{two-sample-kolmogorov-smirnov-test}{%
\subsection{Two-sample Kolmogorov-Smirnov
test}\label{two-sample-kolmogorov-smirnov-test}}

\hypertarget{section-1}{%
\subsection{}\label{section-1}}

\hypertarget{data-penguin_subdatainterstretch_interval_s-and-rpois5000-10}{%
\subsection{data: penguin\_subdata\$Interstretch\_interval\_s and
rpois(5000,
10)}\label{data-penguin_subdatainterstretch_interval_s-and-rpois5000-10}}

\hypertarget{d-0.8116-p-value-2.2e-16}{%
\subsection{D = 0.8116, p-value \textless{}
2.2e-16}\label{d-0.8116-p-value-2.2e-16}}

\hypertarget{alternative-hypothesis-two-sided}{%
\subsection{alternative hypothesis:
two-sided}\label{alternative-hypothesis-two-sided}}

\begin{verbatim}
\end{verbatim}

\hypertarget{section-2}{%
\subsection{}\label{section-2}}

\hypertarget{shapiro-wilk-normality-test}{%
\subsection{Shapiro-Wilk normality
test}\label{shapiro-wilk-normality-test}}

\hypertarget{section-3}{%
\subsection{}\label{section-3}}

\hypertarget{data-penguin_subdatainterstretch_interval_s}{%
\subsection{data:
penguin\_subdata\$Interstretch\_interval\_s}\label{data-penguin_subdatainterstretch_interval_s}}

\hypertarget{w-0.084277-p-value-2.2e-16}{%
\subsection{W = 0.084277, p-value \textless{}
2.2e-16}\label{w-0.084277-p-value-2.2e-16}}

\begin{verbatim}
No, this distribution is decidedly non- normal. 

But is it not normal at each group size? It is possible that each group has it's own inter-stretch interval, and combining all the group sizes makes the distribution appear non-normal when it is. 
histograms broken down by group size
![](penguin_stretching_files/figure-latex/unnamed-chunk-12-1.pdf)<!-- --> ![](penguin_stretching_files/figure-latex/unnamed-chunk-12-2.pdf)<!-- --> ![](penguin_stretching_files/figure-latex/unnamed-chunk-12-3.pdf)<!-- --> 


KS test at each group size



Table: Summary of normality tests by group size

 n_penguins   shapiro_statistic   shapiro_p.value   KStest_statistic   KStest_p.value
-----------  ------------------  ----------------  -----------------  ---------------
          1           0.6306605         0.0000000          0.9929078                0
          2           0.6399587         0.0000000          0.9626899                0
          3           0.6410582         0.0000000          0.9453095                0
          4           0.5712584         0.0000000          0.9376458                0
          5           0.6037727         0.0000000          0.9078143                0
          6           0.4899363         0.0000000          0.8845676                0
          7           0.6080773         0.0000000          0.8951143                0
          8           0.4679993         0.0000000          0.9050434                0
          9           0.6500233         0.0000000          0.9302843                0
         10           0.6164104         0.0000000          0.9257965                0
         11           0.7647816         0.0000000          0.9365000                0
         12           0.8339019         0.0007042          0.9615385                0
This shows that all the group sizes were non-normal, including the observations with only 1 penguin in a group. Suggests that differences in group size may not account for the non-normality of the whole data set. However, groups of 1 were also not normal! 



###Some other explorations

When did people observe?
Did median stretch interval change throughout the day?
Did group median size vary throughout the day?

\end{verbatim}

\hypertarget{late-afternoon-early-afternoon}{%
\subsection{{[}1{]} ``Late afternoon'' ``Early
afternoon''}\label{late-afternoon-early-afternoon}}

\hypertarget{late-morning-night}{%
\subsection{{[}3{]} ``Late morning''
``Night''}\label{late-morning-night}}

\hypertarget{late-morning-early-afternoon}{%
\subsection{{[}5{]} ``late-morning'' ``early
afternoon''}\label{late-morning-early-afternoon}}

\hypertarget{late-afternnon-early-morning}{%
\subsection{{[}7{]} ``late afternnon'' ``Early
morning''}\label{late-afternnon-early-morning}}

\hypertarget{morning-evening}{%
\subsection{{[}9{]} ``morning'' ``evening''}\label{morning-evening}}

\hypertarget{morning-afternoon}{%
\subsection{{[}11{]} ``Morning''
``Afternoon''}\label{morning-afternoon}}

\hypertarget{afternoon-na}{%
\subsection{{[}13{]} ``afternoon'' NA}\label{afternoon-na}}

\hypertarget{moring-late-morning--mehrdad}{%
\subsection{{[}15{]} ``moring'' ``Late morning-
Mehrdad''}\label{moring-late-morning--mehrdad}}

\hypertarget{late-morningearly-afternoon}{%
\subsection{{[}17{]} ``Late morning/Early
afternoon''}\label{late-morningearly-afternoon}}

\begin{verbatim}
\end{verbatim}

\hypertarget{a-tibble-5-x-4}{%
\subsection{\# A tibble: 5 x 4}\label{a-tibble-5-x-4}}

\hypertarget{time_of_day-number_observations-median_intervals-median_groupsize}{%
\subsection{Time\_of\_day number\_observations median\_intervals
median\_groupsize}\label{time_of_day-number_observations-median_intervals-median_groupsize}}

\hypertarget{section-4}{%
\subsection{\texorpdfstring{ }{   }}\label{section-4}}

\hypertarget{early-afternoon-1163-84.0-4}{%
\subsection{1 Early afternoon 1163 84.0
4}\label{early-afternoon-1163-84.0-4}}

\hypertarget{early-morning-548-57.0-7}{%
\subsection{2 Early morning 548 57.0 7}\label{early-morning-548-57.0-7}}

\hypertarget{late-afternoon-1637-69.-5}{%
\subsection{3 Late afternoon 1637 69.
5}\label{late-afternoon-1637-69.-5}}

\hypertarget{late-morning-1473-52.-5}{%
\subsection{4 Late morning 1473 52. 5}\label{late-morning-1473-52.-5}}

\hypertarget{night-1468-59.-5}{%
\subsection{5 Night 1468 59. 5}\label{night-1468-59.-5}}

```

Plot of when people observed:
\includegraphics{penguin_stretching_files/figure-latex/unnamed-chunk-16-1.pdf}

Plot of group size by time of day
\includegraphics{penguin_stretching_files/figure-latex/unnamed-chunk-17-1.pdf}

\end{document}
